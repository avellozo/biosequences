\documentclass[12pt]{article}
\usepackage{latexsym}
\usepackage{amssymb}
\usepackage[brazil]{babel} \usepackage[latin1]{inputenc}
\usepackage[normalem]{ulem} \usepackage[portuguese]{alg}
\usepackage[dvips]{graphicx}
\usepackage{amsmath}
\usepackage{hyperref}

\usepackage{alg}
\usepackage{float}
\floatstyle{ruled}
\newfloat{algorithm}{htbp}{loa}%[chapter]
\newcommand{\listofalgorithms}{\listof{algorithm}{Lista de Algoritmos}}
\floatname{algorithm}{Algoritmo} 
 
% Teoremas, Lemas, etc.
\newtheorem{teorema}{Teorema}[section]
\newtheorem{lema}[teorema]{Lema}
\newtheorem{proposicao}[teorema]{Proposi��o}
\newtheorem{observacao}[teorema]{Observa��o}
\newtheorem{corolario}[teorema]{Corol�rio}
\newtheorem{definicao}[teorema]{Defini��o}
\newtheorem{exemplo}[teorema]{Exemplo}
\newtheorem{problema}[teorema]{Problema}
\newenvironment{prova}[1][Prova.]{\begin{trivlist}
\item[\hskip \labelsep {\itshape #1}]}{\end{trivlist}}
\newenvironment{comentario}[1][Coment�rio]{\begin{trivlist}
\item[\hskip \labelsep {\bfseries #1}]}{\end{trivlist}}


\newcommand{\tq}{\mathrel{|}}
\newcommand{\weightsym}{\phi}
\newcommand{\weight}[1]{\function{\weightsym}{#1}}
\newcommand{\invert}[1]{\mathord{\overline{{#1}}}}
\newcommand{\invertsym}{\mathord{\overline{\ \ {\vspace{3 mm}}}}}
\newcommand{\aplica}{\longrightarrow}              
\newcommand{\implica}{\Longrightarrow}   
\newcommand{\sse}{\Longleftrightarrow}   
\newcommand{\seq}{sequ�ncia}
\newcommand{\seqs}{se\-qu�n\-cias}
           

% \newcount\hour
% \newcount\minute
% \hour=\time \divide \hour by 60 \minute=\time \loop \ifnum \minute
% > 59 \advance \minute by -60 \repeat
% \def\now{%
% \ifnum \hour<13 \ifnum \hour<1 12:\else\number\hour:\fi \ifnum
% \minute<10 0\fi
% \number\minute%
% %\ifnum \hour<12 \ AM \else \ PM \fi
% \else \advance \hour by -12 \number\hour:%
% \ifnum \minute<10 0\fi
% \number\minute%
% %\ PM%
% \fi%
% }
% \let\oldthepage=\thepage
% \renewcommand{\thepage}{\today{} \now \qquad (\oldthepage)}

\newcommand{\cqd}{\nobreak \ifvmode \relax \else
      \ifdim\lastskip<1.5em \hskip-\lastskip
      \hskip1.5em plus0em minus0.5em \fi \nobreak
      \vrule height0.75em width0.5em depth0.25em\fi}

% \usepackage[small,compact]{titlesec}

\newcommand{\squishlist}{
   \begin{list}{$\bullet$}
    { \setlength{\itemsep}{0pt}      \setlength{\parsep}{3pt}
      \setlength{\topsep}{3pt}       \setlength{\partopsep}{0pt}
      \setlength{\leftmargin}{1.5em} \setlength{\labelwidth}{1em}
      \setlength{\labelsep}{0.5em} } }

\newcommand{\squishlisttwo}{
   \begin{list}{$\bullet$}
    { \setlength{\itemsep}{0pt}    \setlength{\parsep}{0pt}
      \setlength{\topsep}{0pt}     \setlength{\partopsep}{0pt}
      \setlength{\leftmargin}{2em} \setlength{\labelwidth}{1.5em}
      \setlength{\labelsep}{0.5em} } }

\newcommand{\squishend}{ \end{list}  }

\begin{document}
\thispagestyle{empty}
Augusto Fernandes Vellozo

10B, rue Paul Cambon

69100 - Villeurbanne 

FRANCE

Telefone: +33(0)426766582

E-mail: augusto@vellozo.org
\vspace{30pt}

\hspace{170pt} Arlindo Manuel Limede de Oliveira

\hspace{170pt} INESC-ID

\hspace{170pt} R Alves Redol 9

\hspace{170pt} 1000 Lisboa, Portugal

\vspace{50pt}

Caro Senhor/Senhora/Senhorita,
\vspace{20pt}

Eu, Augusto Fernandes Vellozo, cidad�o portugu�s com bilhete de identidade
n�mero 15591425-1, venho por meio desta carta candidatar-me para a posi��o de
investigador doutorado no INESC-ID em Lisboa Portugal.

Como foi solicitado, encaminho anexo
a esta carta, meu \emph{curriculum vitae} detalhado, um plano de atividades a
serem desenvolvidas e 3 cartas de refer�ncia.

Atualmente trabalho no laborat�rio BAOBAB coordenado por Marie-France Sagot em
Lyon, onde tive e tenho contato com investigadores portugueses, tais como Ana
Teresa de Freitas. Sei que existe uma forte colabora��o entre o INESC-ID e o
laborat�rio BAOBAB e acredito que possa ajudar bastante em futuras colabora��es
com este laborat�rio.


Tenho especial interesse em trabalhar em Portugal e principalmente em Lisboa,
pois em meu pa�s sei que terei melhores condi��es de adapta��o junto com
minha fam�lia, al�m de usufruir do clima da cidade de Lisboa.

Desde j� agrade�o pela aten��o,

\vspace{50pt}

Augusto Fernandes Vellozo




\end{document}

