\documentclass[10pt]{article}
\usepackage{latexsym}
\usepackage{amssymb}
\usepackage[brazil]{babel} 
\usepackage[latin1]{inputenc} \usepackage[normalem]{ulem} 
\usepackage[dvips]{graphicx}
\usepackage{amsmath}
% \usepackage{subfigure}
%\usepackage{fullpage}


%\usepackage[portuguese]{alg}
%\usepackage{alg}
\usepackage{alg}
%% %% \usepackage[boxed,chapter]{algorithm}
%% %% \renewcommand{\listalgorithmname}{Lista de Algoritmos}
%% %% \floatname{algorithm}{Algoritmo}
\usepackage{float}
%%\floatstyle{boxed}
\floatstyle{ruled}
\newfloat{algorithm}{htbp}{loa}%[chapter]
 
%layout
%\renewcommand{\baselinestretch}{1.7}

% % Para controlar diferentes vers�es
% % getting the time and using it via \now
% \newcount\hour
% \newcount\minute
% \hour=\time \divide \hour by 60 \minute=\time \loop \ifnum \minute
% > 59 \advance \minute by -60 \repeat
% \def\now{%
% \ifnum \hour<13 \ifnum \hour<1 12:\else\number\hour:\fi \ifnum
% \minute<10 0\fi
% \number\minute%
% %\ifnum \hour<12 \ AM \else \ PM \fi
% \else \advance \hour by -12 \number\hour:%
% \ifnum \minute<10 0\fi
% \number\minute%
% %\ PM%
% \fi%
% }
% \let\oldthepage=\thepage
% \renewcommand{\thepage}{\today{} \now \qquad (\oldthepage)}
% 
%Teoremas, Lemas, etc.
\newtheorem{teorema}{Teorema}[section]
\newtheorem{lema}[teorema]{Lemma}
\newtheorem{proposicao}[teorema]{Proposition}
\newtheorem{observacao}[teorema]{Note}
\newtheorem{corolario}[teorema]{Corollary}
\newtheorem{definicao}[teorema]{Definition}
\newtheorem{exemplo}[teorema]{Exemple}
\newtheorem{problema}[teorema]{Problem}

\newenvironment{prova}[1][Proof.]{\begin{trivlist}
\item[\hskip \labelsep {\itshape #1}]}{\end{trivlist}}
% \newenvironment{definicao}[1][Defini��o]{\begin{trivlist}
% \item[\hskip \labelsep {\bfseries #1}]}{\end{trivlist}}
% \newenvironment{problema}[1][Problema]{\begin{trivlist}
% \item[\hskip \labelsep {\bfseries #1}]}{\end{trivlist}}
% % \newenvironment{exemplo}[1][Exemplo]{\begin{trivlist}
% % \item[\hskip \labelsep {\bfseries #1}]}{\end{trivlist}}
\newenvironment{comentario}[1][Comentary]{\begin{trivlist}
\item[\hskip \labelsep {\bfseries #1}]}{\end{trivlist}}

\newcommand{\cqd}{\nobreak \ifvmode \relax \else \ifdim\lastskip<1.5em 
\hskip-\lastskip \hskip1.5em plus0em minus0.5em \fi \nobreak \vrule 
height0.75em width0.5em depth0.25em\fi}
\newcommand{\tq}{\mathrel{|}}
\newcommand{\weightsym}{\phi}
\newcommand{\weight}[1]{\function{\weightsym}{#1}}
\newcommand{\invert}[1]{\mathord{\overline{{#1}}}}
\newcommand{\invertsym}{\mathord{\overline{\ \ {\vspace{3 mm}}}}}
\newcommand{\aplica}{\longrightarrow}              
\newcommand{\implica}{\Longrightarrow}   
\newcommand{\sse}{\Longleftrightarrow}   
           
\DeclareTextFontCommand{\textcourier}{\fontfamily{pcr}\selectfont}

\title{Resumo das publica��es}
\author{Augusto Fernandes Vellozo}

\begin{document}

\maketitle

Neste texto, apresento a seguir o resumo das duas publica��es que tenho at� o momento.

\section{Alignment with non-overlapping inversions in $O(n^3)$-time~\cite{vellozo06:_align_with_non_overl_inver}}

  Alignments of sequences are widely used for biological sequence comparisons. 
  Only biological events like mutations, insertions and deletions are usually 
  modeled and other biological events like inversions are not automatically 
  detected by the usual alignment algorithms.
  
  Alignment with inversions does not have a known polynomial algorithm and a 
  simplification to the problem that considers only non-over\-lapping 
  inversions were proposed by Sch�niger and Waterman~\cite{pmid1591531} in 1992 
  as well as a corresponding $O(n^6)$ solution\footnote{In this case, $n$ 
  denotes the maximal length of the two aligned sequences.}.  An improvement to 
  an algorithm with $O(n^3 \log n)$-time complexity was announced in an 
  extended abstract~\cite{MR2173809} and, in this present paper, we give an 
  algorithm that solves this simplified problem in $O(n^3)$-time and 
  $O(n^2)$-space in the more general framework of an edit graph.

  Inversions have 
  recently~\cite{pmid15746427,pmid10411506,pmid10794175,pmid15466707} been 
  discovered to be very important in Comparative Genomics and Scherer et al.\ 
  in 2005~\cite{pmid16254605} experimentally verified inversions that were 
  found to be polymorphic in the human genome.  Moreover, 10\% of the 1,576 
  putative inversions reported overlap RefSeq genes in the human genome.  We 
  believe our new algorithms may open the possibility to more detailed studies 
  of inversions on DNA sequences using exact optimization algorithms and we 
  hope this may be particularly interesting if applied to regions around known 
  rearrangements boundaries.  Scherer report 29 such cases and prioritize them 
  as candidates for biological and evolutionary studies.
\newpage

\section{Alignment with non-overlapping inversions in $O(n^3 \log n)$-time 
(extended abstract)~\cite{MR2173809}}

\label{sec:Introduction} Alignment of sequences is widely used for biological 
sequence comparisons and can be associated with a set of edit operations that 
transform one sequence to the other.  Usually, the only edit operations that 
are considered are the \emph{substitution} (mutation) of one symbol by another 
one, the \emph{insertion} of one symbol and \emph{deletion} of one symbol.  If 
costs are associated with each operation, there is a classic $O(n^2)$ dynamic 
program\footnote{In this paper, $n$ denotes the maximal length of the two 
aligned sequences.} that computes a set of edit operations with minimal total 
cost and exhibit the associated alignment, which has good quality and high 
likelihood for realistic costs.

Other important biological events like inversions are not automatically 
detected by the usual alignment algorithms and we can define a new edit 
operation, the \emph{inversion} operation, which substitutes any segment by its 
\emph{reverse complement} sequence.  We can define a new alignment problem: 
given two sequences and fixed costs for each kind of edit operation, the 
\emph{alignment with inversions} problem is an optimization problem that 
queries the minimal total cost of an edit operations set that transforms one 
sequence to the other.  Moreover, one may also be interested in the exhibition 
of its correspondent alignment and/or edit operations.  To the best of our 
knowledge, the computational complexities of alignment with inversions problem 
is unknown.

Some simplifications of this problem have been studied and were proved to be 
NP-complete~\cite{Wagner75,MR2000a:68046}.  Many approximation algorithms were 
also proposed~\cite{MR95j:68125,MR1642934}.  Another important simplification 
is the problem known as \emph{sorting signed permutations by reversals} and 
many polynomial algorithms have been 
obtained~\cite{225112,MR2001c:92008,MR2003i:68102}.

Another important approach was introduced in 1992, by Sch�niger and 
Waterman~\cite{pmid1591531}.  They introduced a \emph{simplification 
hypothesis}: \emph{all regions involving in the inversions do not overlap}. 
This led to the \emph{alignment with non-overlapping inversions} problem and 
they presented a $O(n^6)$ solution for this problem and also introduced a 
\emph{heuristic} for it that reduced the running-time to something between 
$O(n^2)$ and $O(n^4)$.

Recently, indepent 
works~\cite{gao03:_space_effic_algor_sequen_align_inver,lago03:wob03,lago01:_compar,MR2132586} 
gave exact algorithms for alignments with non-overlapping inversions with 
$O(n^4)$-time and $O(n^2)$-space complexity.  In this present extended 
abstract, we announce an algorithm that solves this simplified problem in 
$O(n^3 \log n)$-time and $O(n^2)$-space.

\newpage

\bibliographystyle{plain}
\bibliography{projetos}

\end{document}
