\documentclass[12pt]{article}
\pagestyle{headings}
\usepackage{latexsym}
\usepackage{amssymb}
\usepackage[brazil]{babel} 
\usepackage[latin1]{inputenc}
\usepackage[normalem]{ulem} 
\usepackage[portuguese]{alg}
\usepackage{graphicx}
\usepackage{amsmath}

\usepackage{alg}
\usepackage{float}
\floatstyle{ruled}
\newfloat{algorithm}{htbp}{loa}%[chapter]
\newcommand{\listofalgorithms}{\listof{algorithm}{Lista de Algoritmos}}
\floatname{algorithm}{Algoritmo} 
 
%Teoremas, Lemas, etc.
\newtheorem{teorema}{Teorema}[section]
\newtheorem{lema}[teorema]{Lema}
\newtheorem{proposicao}[teorema]{Proposi��o}
\newtheorem{observacao}[teorema]{Observa��o}
\newtheorem{corolario}[teorema]{Corol�rio}
\newtheorem{definicao}[teorema]{Defini��o}
\newtheorem{exemplo}[teorema]{Exemplo}
\newtheorem{problema}[teorema]{Problema}
\newenvironment{prova}[1][Prova.]{\begin{trivlist}
\item[\hskip \labelsep {\itshape #1}]}{\end{trivlist}}
\newenvironment{comentario}[1][Coment�rio]{\begin{trivlist}
\item[\hskip \labelsep {\bfseries #1}]}{\end{trivlist}}

\newcommand{\tq}{\mathrel{|}}
\newcommand{\weightsym}{\phi}
\newcommand{\weight}[1]{\function{\weightsym}{#1}}
\newcommand{\invert}[1]{\mathord{\overline{{#1}}}}
\newcommand{\invertsym}{\mathord{\overline{\ \ {\vspace{3 mm}}}}}
\newcommand{\aplica}{\longrightarrow}              
\newcommand{\implica}{\Longrightarrow}   
\newcommand{\sse}{\Longleftrightarrow}   
\newcommand{\seq}{seq��ncia}
\newcommand{\seqs}{seq��ncias}
           

% \newcount\hour
% \newcount\minute
% \hour=\time \divide \hour by 60 \minute=\time \loop \ifnum \minute
% > 59 \advance \minute by -60 \repeat
% \def\now{%
% \ifnum \hour<13 \ifnum \hour<1 12:\else\number\hour:\fi \ifnum
% \minute<10 0\fi
% \number\minute%
% %\ifnum \hour<12 \ AM \else \ PM \fi
% \else \advance \hour by -12 \number\hour:%
% \ifnum \minute<10 0\fi
% \number\minute%
% %\ PM%
% \fi%
% }
% \let\oldthepage=\thepage
% \renewcommand{\thepage}{\today{} \now \qquad (\oldthepage)}

\newcommand{\cqd}{\nobreak \ifvmode \relax \else
      \ifdim\lastskip<1.5em \hskip-\lastskip
      \hskip1.5em plus0em minus0.5em \fi \nobreak
      \vrule height0.75em width0.5em depth0.25em\fi}

\begin{document}

\title{Documento de conclus�o do doutorado}
\author{Augusto Fernandes Vellozo}
\maketitle O documento mostrado nas p�ginas a seguir foi obtido na secretaria 
de p�s-gradua��o do IME-USP. Neste documento consta que a data limite para o 
t�rmino do doutorado � dia 02/03/2007, portanto antes do in�cio do per�odo de 
vig�ncia da bolsa de p�s-doutorado. Al�m disto, vale a pena ressaltar que 
apesar de n�o constar no documento abaixo, a banca para a defesa do doutorado 
j� foi aprovada em reuni�o da comiss�o de p�s-gradua��o no dia 15 de Dezembro 
de 2006. J� entramos em contato com os professores da banca e foi acertado com 
\textbf{todos} que a data para a defesa ser� dia \textbf{15 de Fevereiro de 
2007}, portanto 2 meses e meio antes do in�cio da bolsa de p�s-doutorado. Estes 
dados n�o constam no documento, pois o documento s� � atualizado ap�s o 
dep�sito da tese, que ocorrer� at� o dia 15 de Janeiro de 2007. \newpage
\begin{figure}
\begin{center}
\includegraphics[scale=0.7]{docIme1.jpg}
\end{center}
\end{figure}
\newpage
\begin{figure}
\begin{center}
\includegraphics[scale=0.7]{docIme2.jpg}
\end{center}
\end{figure}
\newpage
\begin{figure}
\begin{center}
\includegraphics[scale=0.7]{docIme3.jpg}
\end{center}
\end{figure}
\end{document}

