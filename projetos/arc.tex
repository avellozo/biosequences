\documentclass[11pt]{article}
\marginparwidth 0pt
\oddsidemargin  0pt

\marginparsep 0pt

\topmargin -0.25in

\textwidth 6.5in
\textheight 8.5in

\date{}

\usepackage[latin1]{inputenc}
\usepackage{amsmath}
\usepackage{amsfonts}
\usepackage{epsfig}
\usepackage{amssymb}
\usepackage{latexsym}
\usepackage{euscript}
\usepackage{xspace}
\usepackage{mathrsfs}
\setcounter{tocdepth}{1}
%%%%%%%%%%%%%%%%%%%%%%%%%%%%%%%%%%%%%%%%%%%%%%%%%%%%%%%%%%%%%%%%%%%%%%
\begin{document}
%%%%%%%%%%%%%%%%%%%%%%%%%%%%%%%%%%%%%%%%%%%%%%%%%%%%%%%%%%%%%%%%%%%%%%
\title{Action de Recherche Coop�rative de la Direction Scientifique\\
\ \\ \ \\Project title: Integrated Biological Networks\\ 
{\small Coordinator: Marie-France Sagot, HELIX Project}}
%%%%%%%%%%%%%%%%%%%%%%%%%%%%%%%%%%%%%%%%%%%%%%%%%%%%%%%%%%%%%%%%%%%%%%

\maketitle

\section{Research project}

%MF: ajouter references, les notres et celles d'autres (par rapport 
%    au projet de recherche.

Historically, the field of computational biology, at least from the
more algorithmic point of view, started with the analysis of
sequences, isolated sequences for many years, and then whole genomes
when the sequencing projects started accelerating. Other topics of
interest in the early years of the discipline addressed issues that
were either sequence-related (for instance, molecular phylogeny) or
were related to the analysis of other elementary biological objects
(such as protein or RNA structures).

More recently, concern with getting a deeper understanding on how all
such elementary biological objects interact in the general context of
a genome, cell or organism has led to the development of whole new
areas of investigation by computational biologists. The study of
relations, although not altogether new to the field, has thus been in
full bloom in the last few years. Such relations concern {\it every}
element in a cell or organism. They may even concern extra-cellular
elements, or elements that belong to the environment of an organism as
some may have an influence on the inner functioning of a living
system. Investigating relations requires therefore to study what has
been called {\bf integrative biology}. This is the objective of the
present project.

This project thus fits the first of the ``Favored directions'' for
2005, namely ``Coupling models and data to simulate and control
complex systems'' although our main aim is first to understand such
complex systems, with simulation as a second, more longer term
objective at least as bigger systems are concerned (simulation of
smaller systems is already being studied in one of the teams).

\vspace{0.5cm} Lack of enough or sufficiently clean data has however
hindered for now a full growth of the areas of investigation within or
related to integrative biology. Worse still, the lack of good models
has slowed down the development of revolutionary new ways of
considering such relations, and thus of considering biology at the
global level of an organism, in a way similar to the revolution
brought about by the sequencing of genomic sequences to our view of
genetics and molecular evolution.  The data come under many different
forms, whose nature falls into three main categories: 1. sequence
data, 2. expression data (this includes both transcriptomic and
proteomic data), and 3. biochemical data (such as provided by the
lists of reactions, enzymes and compounds composing the metabolism of
an organism). This diversity requires expertise in different types of
techniques such as text algorithmics, combinatorial and statistical
data analysis methods, and graph theory modelling and algorithmics.

We wish to address both types of problems in this project: lack of
reliable data/information (on actually existing relations) and lack of
good models (in particular for what could be defined as conserved, and
therefore potentially functional, relations). The diversity of the
forms of data to be treated means already that the models must vary as
well.  As a reflection of this, the present project is divided into
three main deeply inter-related topics of investigation: exploration
and analysis of the complex regulation motifs that represent important
elements in any study of biochemical and evolutionary networks (Part 1
of this project), genome dynamics (Part 2), and genetic and
biochemical networks (Part 3).

Like many other issues in computational biology, obtaining good models
is a particularly difficult problem both because we lack general
knowledge of the biological processes at play (and so have no clear
idea of where to look at, or even what we are supposed to look for),
and because we have at our disposal an often vast collection of
partial and very specialized knowledge (that may strongly bias our
search for new information, or even prevent us completely from finding
anything new).

The investigation for good models (it is very unlikely that a unique
one will exist) must therefore involve all the steps from
1. biological theory (of what could be the important forces or
constraints to consider) to 2. the development of mathematical models
that lead to 3. (theoretical) considerations of algorithmical
complexity which may either 4. feed back into the biological
formulations of the problems considered or 4'. lead to the elaboration
of efficient algorithms, and then 5. to the application of such
algorithms to data in a specific or systematic way, 6. analysis of the
results obtained and 7. feedback to either the initial theory or the
mathematical models derived from it. Upstream of this, the problem of
obtaining the elementary pieces of information (on what relations
actually exist) must be addressed.

In the context of this project, we shall concentrate our attention on
a still somewhat restricted view of the functioning of a cell or
organism, the one provided by biochemical and evolutionary networks,
and by the relation between the two. A biochemical network offers a
view that is in-between the one given by a molecular sequence and
structure, and the view provided by looking at a cell
directly. Genetic and protein-protein networks can both be seen as
networks of interactions between objects (genes, proteins, sometimes
metabolites) whose deeper nature is not important in a first
approximation. This brings them closer to other types of networks
(social, web) that have been studied for a longer time. Although the
rapprochement is dubious, this has meant that many pre-existing
algorithmic and statistical techniques could be applied in a first
investigation of such networks. Inferring the networks in a reliable
way remains however very much an open problem. Immediate transposition
of the techniques used for studying interaction graphs to metabolic
networks is simply not possible. It is indeed imperative in this case
to take into account the nature both of the elementary objects
represented by the nodes of the network and of the links between them
when modelling and analysing such networks. This in turn requires new
algorithmic and mathematical developments.

The challenge after that will then be to consider the relations
existing between the networks themselves, either by trying to
integrate the various informations they contain into a single network,
or by going one step higher in our view of the functioning of a cell
and building a network of networks. At some point, connection with
genome architecture and organisation, and thus evolutionary networks
becomes essential. It has been relatively little explored so far
although many groups in the world are getting very active in this
area.

\vspace{0.5cm} The teams associated with this project, both INRIA and
external (two INRA groups and a group from the Pasteur Institute),
have all been pursuing research around the topics sketched above for
some years now. Among the teams, there is a mix of groups who have
been in contact for some years (SYMBIOSE project, INRA Toulouse,
Jouy-en-Josas and InaPG Paris) and of groups whose interaction is much
more recent (Systems Biology Unit at the Pasteur Institute, MISTIS
project). In the two cases, the contact has often been of a loose
nature not formalized through a common project. The teams have both
overlapping and complementary concerns and expertises.  The teams thus
gather people with interest in and/or specialized in algorithmics,
combinatorial optimization and graph theory, statistics, and
computational biology. Among them, all the main subjects to be covered
by the project are currently been addressed: protein-protein and
genetic interaction networks (modelling, inference and simulation),
metabolic networks (inference and analysis), sequence and structure
motifs involved in regulation (modelling, search and inference),
evolutionary processes (deriving synteny relations, studying genome
structure and dynamics and the relation of these with biochemical
networks). Each of the topics detailed below is thus already the
object of a research that is being conducted in at least one of the
teams, either individually or in collaboration. Sharing knowledge
within these areas is absolutely essential for trying to get at a
bigger picture of the functioning of a cell.

\ \\
\noindent
{\large \bf Part 1: Complex regulation motifs}

The main objective of this part is to address various aspects of the
problem of sequence motif modelling, search and inference. The motifs
that will concern us will be all motifs involved in some way with
regulation. The sequences may be DNA or RNA. Indeed, it is now known
that RNA plays a much more prominent role in regulation than believed
until recently. We thus know now that there is massive transcription
of non-coding RNA. RNA mediated regulation works at two levels, a
digital, message passing level, and an analog, catalytic
level. Current tools are inadequate to explore this ``new world'' of
RNAs.  Even with the long-practiced approach to structure prediction,
a breakthrough is needed to extract the significant ensemble of
structures from the large, near-optimal folding space.  We also need
methods for large-scale search for RNA genes, hybridization sites and
structural patterns based on thermodynamics.

The following subtopics will more particularly interest us in the
context of this ARC project:

\begin{description}
\item[1.1] Improving or investigating the various models for motifs
that have been developed over the years. The motifs here include:
complex motifs, that is motifs which may be composed of various parts
separated by more or less constrained distances; footprint motifs,
that is motifs that explicitely take a (known) phylogeny into account
when trying to infer motifs from a set of orthologous sequences;
finally, motifs that try to combine both the pattern and position
weight matrix models for representing sequence signals.

\item[1.2] Systematically investigating new approaches for motif
inference, in particular by exploring probabilistic methods and
linking them with combinatorial approaches to motif extraction.

\item[1.3] Exploiting conservation differences, that is, differences
in evolutionary constraints, to identify gene regulatory
sequences. Algorithms will be considered for scanning sequences from
different organisms for conserved equivalent segments, and within
these segments, to identify hyper conserved and hyper variable
regions.

\item[1.4] Analysing various exact or approximate formulations of
motif clustering and improving the algorithms for establishing the
statistical significance of any type of sequence motifs.

\item[1.5] (This overlaps with Part 3 of the current project.)
Developing methods for the inference of regulatory networks by
combining information from two sources: microarray genomic expression
data (using for instance biclustering algorithms) and genetic sequence
data.

\item[1.6] Locating RNAs (genes and motifs) in genomic
sequences. Two main approaches will be developed:

\begin{itemize}
\item[a.] Inferring RNA secondary structures from unaligned sequences
of homologous RNA genes. The final objective is to propose a method
enabling to infer a searchable description of a common motif.

\item[b.] Locating unknown RNA genes in genomic sequences. The
objective in this case is to improve the detection of RNA genes by
merging several different types of information.

\item[c.] Developing techniques to evaluate the significance of
putative RNA motifs.
\end{itemize}

\end{description}

\ \\
\noindent
{\large \bf Part 2: Genome dynamics}

It has since long been known that genomes are not static. The work of
Barbara Clintock in the late 40's showing that genes could jump
spontaneously from one site to another was a first clear sign of
this. ``Jumping genes'' were called transposable elements by
Clintock. Genes may also get duplicated. There are strong indications
that the duplication may sometimes affect whole chromosomes or even
genomes or, inversely, only pieces of a gene, in particular exons. It
has thus been shown that, during evolution, DNA segments coding for
modules or domains in proteins have been duplicated and rearranged
through what has been called intronic recombination. By shuffling
modules between genes, protein families have thus evolved. Genomic
segments can be reversed, in general through ectopic recombination, or
deleted. Chromosomes in multi-chromosomal organisms may undergo fusion
or fission, or exchange genetic material with another chromosome
(through homologous or non-homologous recombination), usually at their
ends (translocation) or internally. Genetic material may also be
transferred across sub-species or species (lateral transfer), thus
leading to the insertion of new elements in a genome.  Parts of a
genome may be amplified, through, for instance, slippage resulting in
the multiplication of the copies of a tandem repeat.

Although much is known about the dynamic behaviour of genomes, much
more remains to be discovered about the forces and exact mechanisms
behind such dynamics, its function and extend, the frequency of each
type of rearrangement, and the impact genomic reorganisations may have
on gene expression and genome development.

The main topics around which the association between the researchers
involved this ARC project can yield breakthroughs are:

\begin{description}
\item[2.1] Algorithms and complexity analysis for calculating
a rearrangement distance between two or more genomes under various
models. Classical methods of DNA sequence comparison assumed that
sequences may only mutate by operations that act on individual
nucleotides, {\it i.e.}, substitutions, insertions, and
deletions. More recently, additional studies considered large scale
genome rearrangement events such as inversions, transpositions and
translocation.  We aim to broaden the theory of genome rearrangement
in several directions, and to tighten the contact between the
theoretical analysis and the real data that are gradually becoming
available. The key topics we shall study are algorithms for sorting by
signed reversals, length-sensitive sorting by reversal, sorting by
transpositions, handling duplicated genes, handling missing genes,
handling multiple genomes.

\item[2.2] Modelling, detection and analysis of ``segments conserved
by rearrangements''. ``Segments conserved by rearrangements'' mean
parts of a chromosome which are relatively stable under large-scale
evolutionary events.  Looking for such segments is a difficult task
for two reasons. The first one is that the conservation is not
exact. Some rearrangements preserve the function associated with a
segment provided there are ``not many'' of them. How precisely to
define the number and type that should be allowed, and therefore which
definition(s) to adopt for a conserved segment (possibly there will be
more than one depending on the biological question) remains very much
an open problem. Our first task will thus be to derive models that are
satisfying both mathematically and biologically. The second difficulty
of the problem is that such models may be hard to compute. Finally,
the links between ``segments conserved by rearrangements'' and
biochemical networks will be explored.

\item[2.3] Study of breakpoint regions of the genome. This consists in
analysing the regions where rearrangements have broken the genome, and
trying to find some characterics that may enable to classify them
according to the type of rearrangement that gave them origin. The
characteristics sought could be the motifs or repeats such regions may
contain, or some other features still to be determined.  We intend to
build methods for detecting such regions as accurately as possible
(this is the counterpart of the conserved segments mentioned just
above). Then by studying the distribution and length of these regions,
we shall try to evaluate the reality of the ``fragile regions'' model,
which asserts that there are evolution hotspots in the genome. This
theory is under discussion in the scientific community, and still
lacks clear theoretical bases. We shall combine gene homology data and
global genomic alignment data and provide reliable tools and analyses
based on previous studies on rearrangements.

\item[2.4] We have interest also in a particular subproblem of the
previous one, namely the problem of alignments with inversions.
Sequence alignments are broadly studied for biological sequence
comparison but considering only biological events such as mutations,
insertions and deletions. Other biological events such as inversions
are not automatically detected by the usual alignment algorithms. Some
alternative strategies have been considered in the attempt to include
inversions and other types of rearrangements.  We plan to improve
further on some initial results that have already been published in
the community concerning this topic, and possibly to generalize them
to other types of alignments and objective functions.

\item[2.5] Repetitions, recombinations and rearrangements. The
objective is to design algorithms for identifying various types of
repeats and studying the relationship between repeats and
recombination on the one hand, and repeats and regulation at the level
of a whole genome on the other.  We shall investigate new models,
algorithms and indexes for identifying various types of repeats in a
sequence. The work will start by attempting a typology of the various
types of repeats that may be found in biological
sequences. Mathematical models and efficient algorithms for their
detection will then be investigated for some of these repeats.

\item[2.6] Exploring the network (graph) nature of the evolution of
organisms.
\end{description}


\ \\
\noindent
{\large \bf Part 3: Genetic and biochemical networks }

It is now commonly accepted that the functioning and development of a
living organism is controlled by the networks of interactions between
its genes, proteins, and small molecules. Studying such networks and
their underlying complexity is the main objective of this part.  This
objective hides a second one, no less crucial, which is to greatly
improve the mathematical and algorithmic theory needed to accurately
model, and then explore and analyse highly complex living
systems. Biochemical networks may represent protein-protein
interactions, the metabolism of an organism, its system of gene
expression regulation, or even, mixed networks that contain
information coming from various of the previous sources.

The amount and spread of the data now becoming available enable us
also to introduce an evolutionary perspective into the study of living
organisms, and in particular of biochemical networks. Evolution is a
general underlying principle of life that allows us to compare and
decipher the meaning and function of structure, the modification of
biochemical pathways and networks, the preservation and variation of
cell signalling systems, and so on. It thus serves to study the
fundamental aspects of life, taking advantage for this of the
exploratory and comparative possibilities provided, in particular, by
the availability of an increasing number of whole sequences and
datasets from different genomes.

We shall be concerned with the following main topics.

\begin{description}

\item[3.1] Proteomics and transcriptomics. The concern will be with
improving the quality of the information that may be derived directly
from both proteomic data (mass spectrometry) and transcriptomic data
(mainly from macro/micro arrays but also from SAGE experiments and EST
analyses). Such information concerns protein identification,
protein-protein interaction and gene co-regulation.

\item[3.2] Motifs and modules in biochemical networks. Modules are in
general considered to be parts of a network that function in relative
independence from other parts, while motifs are small patterns of
interactions that are repeatedly found in the network.  No fully
satisfying or complete definition of motifs and modules in biochemical
networks exist and most of the work will consist in exploring the
various which may be considered (topological or other) and the
algorithmic complexity of such definitions. For each, efficient data
structures, filters and algorithms for both searching known motifs and
for inferring new ones in large networks will be developed.  The
definitions will of course vary depending on the type of biochemical
network that is considered.

The question of the statistical significance of the motifs identified
will be of primary importance. This question is still open. An answer
to it may depend on the definition of a random graph that is
appropriate to the biological problem at hand, a definition of a motif
occurrence in such a network, and how to calculate the probability of
such motifs. The possibility to transpose questions and results on
motif statistics in a random sequence to motif statistics in a random
network will be examined. This will be a more exploratory research
activity.

\item[3.3] Reconstruction, alignment and simulation of metabolic
pathways. Metabolic pathways reflect the sum of an organism's chemical
reactions, and their elucidation is key to the understanding of
cellular processes as a whole.  Such pathways can be represented as
labeled graphs and networks of processes, thus making them amenable to
algorithmic analyses of several kinds.  Our objective is to combine
methods for computational analysis and simulation of these structures
with experimental work that reveals the (kinetic and other) parameters
that are required to characterize the behaviour of these systems in
order to allow life science researchers to better understand how
metabolic pathways function.

In our work, we aim to provide researchers with systematic and
predictive means to do their work.  These include the ability to
compare metabolisms both of a variety of organisms as well as of
similar processes within the same organism, the provision of tools and
methods to do both static and dynamic analyses of pathways, and the
ability to reconstruct complex pathways from their constituents.  Note
that some of the methods developed in this context are applicable also
to other cases, such as regulatory networks, or protein-protein
interaction networks.

\item[3.4] Exploring the relations between biochemical networks (in
particular modules and motifs in such networks) and genome structure
and dynamics (in particular, ``conserved segments'').
 
\item[3.5] Exploring the issues behind biochemical network
integration.

\end{description}

\section{Participants}

The participants to this project comprise three teams from the Inria,
and three external, two from the INRA (Toulouse and
Jouy-en-Josas/Paris) and one from the Pasteur Institute.

In all cases, the lists of participants for each team do not include
students who may arrive at (or return to) the team during the duration
of the project.

\subsection{HELIX, Inria Rh�ne-Alpes - Coordinator team}

The HELIX team is located in Montbonnot (Grenoble) and on the Campus
of the DOUA (Villeurbanne, next to Lyon). The members of the group in
Grenoble, headed by Fran�ois Rechenmann, work in the Rh�ne-Alpes
research unit of the INRIA. The members in Lyon are part of two groups
within the ``Laboratory of Biometry and Biological Evolution'' (LBBE
-- CNRS/Universit� Claude Bernard de Lyon, UMR 5558), directed by
Christian Gautier: the group ``Bioinformatics and Evolutionary
Genomics'' headed by Manolo Gouy, and the group ``BAOBAB'' created in
September 2004 and headed by Marie-France Sagot. The SwissProt group,
headed by Amos Bairoch within the SIB (Swiss Institute of
Bioinformatics) in Geneva, is associated with the HELIX project.
HELIX is composed of computer scientists, mathematicians and
biologists (bio-mathematicians and bio-informaticians).

The objectives of the HELIX team are the design and development of
methods and tools for modelling and analyzing genomic data. Such data
comprise, in addition to biological sequences, a diversity of other
experimentally-obtained information. In all instances, the research
activities of the team are dictated by biological questions, and the
results, in the form of algorithms and computer tools, are evaluated
as to their relevance in computer science or mathematics, and in
biology.

Among the main topics covered by the team are (DNA/RNA and protein)
sequence and structure analysis, comparative genomics and evolution,
computational proteomics and transcriptomics, modelling and analysis
of metabolism, modelling and simulation of genetic regulatory
networks.

\ \\
\noindent
{\bf HELIX participants to the ARC} 

The HELIX participants to the ARC include members from both Grenoble
and Lyon. The coordinator of the ARC is in Lyon.

\begin{itemize}
%MF: Anne?
\item HELIX Grenoble: Fran�ois Rechenmann (head of HELIX), Alain
Viari, Hidde de Jong, Eric Coissac, Michel Page, Gr�gory Batt,
Delphine Ropers, Samuel Drulhe.
\item HELIX Lyon: Marie-France Sagot (coordinator of the ARC project
and head of the BAOBAB team at Lyon), Christian Gautier (head of the
LBBE at Lyon), Eric Tannier, Laurent Gu�guen, Fr�d�ric Boyer, Vincent
Navratil, Christelle Melo de Lima, Leonor Palmeira, Vincent Lacroix,
Claire Lema�tre.
\end{itemize}

Will also participate the following external PhD students currently
co-supervised by a member of HELIX: Julien Allali, Pierre Peterlongo
and Beno�t Olivieri from the University of Marne-la-Vall�e and
Alexandra Carvalho from the Instituto Superior T�cnico of Lisbon,
Portugal.

\ \\
\noindent
{\bf WEB page of HELIX:}

www-helix.inrialpes.fr

\subsection{SYMBIOSE, Inria Rennes}

The SYMBIOSE project is interested in two types of data: sequences
(DNA -- genomes or SNP data -- RNA or proteins) coming from public
databanks and experimental data generated from post-genomic
studies. The first type may be represented with words on a finite
alphabet (4 to 20 letters). For the second one, raw data are images
corresponding to expression levels of genes or mass spectra of
proteins. For expression levels, the current technology offers mostly
qualitative data.

The research specificities of SYMBIOSE include an interest in large
scale studies (genomes or proteomes) and in pattern discovery methods
on sets of sequences. Two main tracks are studied: modelling with
formal languages and development of dedicated machines. Other emerging
or more transversal topics include gene networks modelling and
classification.

\ \\
\noindent
{\bf SYMBIOSE participants to the ARC} 

The participants of SYMBIOSE to the project are Jacques Nicolas
(head of SYMBIOSE), Michel Le Borgne, Anne Siegel, Fran�ois Coste,
Dominique Lavenier.

\ \\
\noindent
{\bf WEB page of SYMBIOSE:}

http://www.irisa.fr/symbiose/index-eng.htm

\subsection{MISTIS, Inria Rh�ne-Alpes}

The aim of MISTIS is to develop statistical methods for applications
in image analysis and signal processing, and particularly for
biological and medical applications. The main research topics of
MISTIS are mixture models, Markovian models and non-parametric
methods.  In particular, research is mainly focused on two directions
of research:

\begin{itemize}
\vspace{-0.2cm}
\item
How to deal with complex phenomena, complex models and complex
data. The team proposes to use structured models and methods allowing
easy interpretations, to develop model selection and approximation
techniques for complex structure models and to study dimension
reduction techniques based on non linear data analysis.
\vspace{-0.2cm}
\item 
The theoretical and practical behaviour of methods. The team's
interest are on approximation justifications, asymptotic behaviour and
convergence analysis.
\end{itemize}

\noindent
{\bf MISTIS participants to the ARC}

The participants of MISTIS to the project are Florence Forbes
(head of MISTIS) and Matthieu Vignes.

\ \\
\noindent
{\bf WEB page of MISTIS:}

http://www.inria.fr/recherche/equipes/mistis.en.html

\subsection{Systems Biology Unit, Pasteur Institute}

The Systems Biology Unit at the Pasteur Institute has been created in
April 2004 with Benno Schwikowski at its head.  The research interests
of the team include:

\begin{itemize}
\vspace{-0.2cm}
\item Qualitative, semi-quantitative and quantitative models for
living systems on the cellular level through integration of diverse
data types;
\vspace{-0.2cm}
\item Computational support of high-throughput technology, such as DNA
arrays and mass spectrometry;
\vspace{-0.2cm}
\item Algorithms.
\end{itemize}

\noindent
{\bf Systems Biology Unit participants to the ARC}

The participants are Benno Schwikowski (head of the Unit) and the PhD
and postdoc students he has just or is currently recruiting.

\ \\
\noindent
{\bf WEB page of Benno Schwikowski:}
http://www.gmd.de/People/Benno.Schwikowski/

\subsection{INRA Toulouse}

The participants to this project from the INRA Toulouse gather members
from two different teams. One is the group ``Statistique et
informatique appliqu�es � la g�n�tique et � la biologie mol�culaire''
that is inside the BIA (``Unit� de Biom�trie et Intelligence
Artificielle'') of the INRA. One of the participants (Thomas Faraut)
belongs to the ``Laboratoire de G�n�tique Cellulaire''.

The two groups are composed of scientists and engineers with expertise
in bioinformatics, computer science, statistics and artificial
intelligence. The aim of the bioinformatics team is to develop
software tools as well as statistical and algorithmic methods to
locate functional elements in genomes and to point out relations
between them. At the genetic marker level, this involves both genetic
and radiated hybrid mapping (with a current specific interest to
comparative mapping and how genome rearrangement analysis may
contribute to a better mapping) and QTL (quantified trait loci)
mapping. At the genomic sequence level, this involves both protein
gene identification and RNA gene structure modelling and search
(either known families or {\it ab initio}). The group applies and
develops expertise in mathematical modelling, constraint satisfaction,
dynamic programming, graph theory, statistical inference, mixture
models in order to extend and build tools and methods well-suited to
genome analysis.

\ \\
\noindent
{\bf INRA Toulouse participants to the ARC}

The participants are Thomas Schiex (head of the group ``Statistique et
informatique appliqu�es � la g�n�tique et � la biologie
mol�culaire''), Christine Gaspin, Thomas Faraut, Simon de Givry,
Matthias Zytnicky and S�verine B�rard.

\ \\
\noindent
{\bf WEB pages of the two groups from the INRA Toulouse:}
http://www.inra.fr/bia/T/schiex/ Genome/ and
http://www.toulouse.inra.fr/lgc/lgc.html

\subsection{SSB, INRA}

The SSB (``Statistique des S�quences Biologiques'') team is mainly
part of the Departement of ``Mathematics and Computer science for the
life sciences'' of the INRA. The group is headed by Sophie Schbath.

The research of the SSB concerns mainly the following topics:

\begin{itemize}
\vspace{-0.2cm}
\item DNA motifs with unexpected frequency in DNA sequences;
\vspace{-0.2cm}
\item DNA motifs unexpectedly distributed along DNA sequences;
\vspace{-0.2cm}
\item Homogeneity of DNA sequences;
\vspace{-0.2cm}
\item Physical mapping;
\vspace{-0.2cm}
\item Score statistics for the analysis and comparison of biological
sequences;
\vspace{-0.2cm}
\item Analysis of microarray data.
\end{itemize}

\noindent
{\bf SSB participants to the ARC}

The participants are Sophie Schbath (head of SSB), St�phane Robin,
Jean-Jacques Daudin, Fran�ois Rodolphe, Catherine Matias, Pierre
Nicolas.

\ \\
\noindent
{\bf WEB page of SSB:} http://www-mig.jouy.inra.fr/ssb/


\section{Budget requested}

We are requesting funds for three main items:

\begin{itemize}
\item two postdocs of one year each, one for the first year and one
for the second year. Besides needed for accelerating the research
covered by the various topics in this project, the postdocs will also
have the important function of effectively strengthening the links
among the different teams. This will be implemented in concrete terms
by having both postdocs travel around and spend some time in the
different teams.  Funds for one machine is also asked for this
position. Since the two postdocs will not overlap in time, a single
machine is sufficient. To facilitate travelling among the teams, a
laptop would be preferred. The cost for each year of postdoc is based
on the rate usually applied by the INRIA.
\item organising one seminar for each year of duration of the project.
Each seminar will consist in a number of talks over a period of 2 or 3
days. The talks will include talks by the team members, and talks of
researchers external to the team that will be invited for the occasion
among international experts in the various topics covered by the
project. The cost of each seminar was calculated based on the cost of
the Seminar Algorithms and Biology
(www.inrialpes.fr/helix/people/sagot/AlgoBio) that the coordinator of
this project has been organising since 1997 at a rate of 2 to 4 per
year, each seminar lasting in general 3 days and including around 15
invited talks from speakers, some from France and most from Europe or
from the United States, particularly the East Coast for cost
reasons. The estimated cost of each such 3-day series of talks of the
Seminar Algorithms and Biology is 15000 euros.  We are requesting two
thirds of that amount for each of the two seminars to be organised
inside this project.
\item covering the expenses of visits among members of the project.
This cost was calculated based on a number of visits per year that
includes:
\begin{itemize}
\item 4 longer term visits of around 2 weeks each;
\item 20 shorter term visits of around 2 days each.
\end{itemize}
This leads to a total of around 68 days of visits (researchers and
students) and 24 train or plane trips per year. We estimated the cost
of a visit at around 70 euros per day (lodging plus meals) and the
cost of a trip at 100 euros on average. This leads to an approximate
total of 7000 euros per year for visits.
\end{itemize}

The total amount of funds requested is summarised in the table below.

\begin{center}
\begin{tabular}{l|l|l|l}
\hline
\multicolumn{4}{c}{{\bf Summary of the funds requested}}\\
\hline
&First year&Second year&Total\\
&(in euros)&(in euros)&(in euros)\\
\hline
Postdocs&36500&36500&73000\\
Equipment for the postdocs&3000&&3000\\
Seminars&10000&10000&20000\\
Visits among teams&7000&7000&14000\\
\hline
Total&56500&53500&110000\\
\hline
\hline
\multicolumn{3}{l|}{{\bf Total requested to the ARC}}&{{\bf 110000}}\\
\end{tabular}
\end{center}

The budget each INRIA team had for 2004 or requested for 2005
is as follows:

\begin{itemize}
\vspace{-0.2cm}
\item HELIX: requested for 2005, 199462 euros;
\vspace{-0.2cm}
\item SYMBIOSE: had for 2004, 148500 euros personel excluded;
\vspace{-0.2cm}
\item MISTIS: requested for 2005, 36700 euros.
\end{itemize}

\end{document}

