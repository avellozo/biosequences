\section{Introdução}
\label{sec:aligndupintro}

Neste capítulo, consideraremos que $s$ e $t$ são duas seqüências de comprimentos 
$n$ e $m$, respectivamente.

Os eventos biológicos típicos que são considerados normalmente nos 
procedimentos de alinhamentos atuais de seqüências de DNA são: substituição, 
remoção e inserção de nucleotídeos. Um outro evento biológico que ocorre, mas 
que normalmente não é considerado nos alinhamentos usuais, é a duplicação, a qual 
iremos considerar sob algumas restrições nos algoritmos de alinhamentos deste 
capítulo. São dois os tipos de duplicações que consideraremos que ocorrem:
duplicações encadeadas (em \emph{tandem}) ou transposições.

Sejam duas seqüências tais que somente uma delas sofreu um evento de duplicação. Ao 
analisarmos estas seqüências através de um alinhamento usual, é muito provável que 
neste alinhamento haja uma seqüência de colunas que correspondam a inserções (ou 
remoções) no trecho que corresponde a duplicação. Neste caso, pretendemos mostrar, 
num alinhamento com duplicações, que houve uma duplicação em uma das seqüências, ao invés de 
mostrar a seqüência de colunas que correspondem a inserções (ou remoções) como é 
mostrado num alinhamento comum.

Em 1997, Benson~\cite{267526} propôs um modelo para o alinhamento de seqüências que 
considera a presença de repetições em \emph{tandem} de mesmo tamanho nas seqüências. 
Ele propôs dois algoritmos exatos para obter um tal alinhamento ótimo que 
considera estas repetições. O primeiro algoritmo proposto executa em tempo 
$O(n^5)$ e espaço $O(n^2)$. O segundo algoritmo proposto executa em tempo 
$O(n^4)$ e espaço $O(n^3)$.

O modelo proposto por 
Benson para considerar duplicações em \emph{tandem}, além da restrição de só 
considerar repetições em \emph{tandem}, tem mais duas restrições, que são as 
seguintes.
\begin{enumerate}
  \item Sejam $s$ e $t$ as seqüências a serem alinhadas. Se um trecho de $s$ é uma 
  repetição então a seqüência base a ser comparada (ou alinhada) com esta repetição em $s$ 
  deve ser um trecho de $t$. Portanto este modelo proposto por Benson não 
  considera que a seqüência original da repetição pode estar na própria seqüência onde está 
  a repetição.
  
  \item A seqüência base deve ser a mesma para repetições consecutivas, ou seja, 
  como as duplicações consideradas são em \emph{tandem} e a seqüência original 
  considerada está sempre na outra seqüência, as seqüências base das duplicações 
  consecutivas devem ter o mesmo tamanho.
\end{enumerate}

Iremos propor modelos parecidos com este modelo apresentado por Benson, porém sem
estas restrições propostas por Benson. Proporemos 
algoritmos exatos que obtêm alinhamentos ótimos com duplicações e
que executam em tempo $O(n^3)$ e memória $O(n^2)$ segundo o nosso modelo.
Utilizando
as técnicas utilizadas no algoritmo $O(n^3)$ para a obtenção de um alinhamento
ótimo com inversões não sobrepostas, existe um algoritmo que, em tempo $O(n^3)$ e memória
$O(n^2)$, obtém um alinhamento ótimo com duplicações em \emph{tandem} segundo o mesmo
modelo proposto pode Benson. Porém acreditamos que os modelos que iremos propor
são modelos mais gerais e que consideram mais casos de duplicações. 
